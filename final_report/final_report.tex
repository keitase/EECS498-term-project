\documentclass{acm_proc_article-sp}

\begin{document}

\title{Determining Community Interest in\\Text-Only Posts on Reddit}
\subtitle{Project Proposal}

%
% You need the command \numberofauthors to handle the 'placement
% and alignment' of the authors beneath the title.
%
% For aesthetic reasons, we recommend 'three authors at a time'
% i.e. three 'name/affiliation blocks' be placed beneath the title.
%

\numberofauthors{3}

\author{
% You can go ahead and credit any number of authors here,
% e.g. one 'row of three' or two rows (consisting of one row of three
% and a second row of one, two or three).
%
% The command \alignauthor (no curly braces needed) should
% precede each author name, affiliation/snail-mail address and
% e-mail address. Additionally, tag each line of
% affiliation/address with \affaddr, and tag the
% e-mail address with \email.
%
% 1st. author
\alignauthor
Charles Lewis\\
       \affaddr{University of Michigan}\\
       \affaddr{Ann Arbor, MI}\\
       \email{noodle@umich.edu}
% 2nd. author
\alignauthor
Nathaniel Price\\
       \affaddr{University of Michigan}\\
       \affaddr{Ann Arbor, MI}\\
       \email{nrprice@umich.edu}
% 3rd. author
\alignauthor
David Purser\\
       \affaddr{University of Michigan}\\
       \affaddr{Ann Arbor, MI}\\
       \email{dpurser@umich.edu}
}

\date{21 April 2014}

\maketitle
\begin{abstract}
This paper proposes a term project for the EECS 498 Information Retrieval
course at the University of Michigan which seeks to analyze content on 
Reddit\footnote{\texttt{http://www.reddit.com/}} to determine community
interest in certain topics, keywords, and questions.

The project involves collecting a set of data from a number of subreddits
(individual forums on Reddit) including textual content of each post,
number of votes that the post received, number of comments on the post, and
other information.  This information will be used to estimate the community's
interest in each post.  These estimates will be compiled and indexed by keyword,
allowing a user to query the system to determine the expected interest in a
new post and whether a new post is very similar to previous posts.
\end{abstract}

% A category with the (minimum) three required fields
\category{H.3.1}{Information Storage and Retrieval}{Content Analysis and Indexing}
\category{H.3.3}{Information Storage and Retrieval}{Information Search and Retrieval}
%A category including the fourth, optional field follows...
%\category{D.2.8}{Software Engineering}{Metrics}[complexity measures, performance measures]

%\terms{Algorithms, }

\keywords{Data analysis, Reddit, community interest} % NOT required for Proceedings

\section{Problem}
We sought to explore whether it was possible to predict whether a post would be popular (community interest) or not on \texttt{www.reddit.com}, specifically text-only posts on subreddits sucht as \texttt{/r/AskReddit} or \texttt{/r/ELI5}. \texttt{Reddit} is a platform where users upvote and downvote posts based on various criteria. The users also have the option of posting a comment on a particular post. The overall motivation was to research different machine learning and information retrieval process in order to figure out which would prove to be helpful in determining community interest on \texttt{Reddit}.

There are various applicaitons in which this work would benefit. This could help indicate when it might be a useful time to repost something. The community of Reddit has a strong aversion to things that are posted often. These are often refered to ``reposts''. However, there is some benefit to reposts. Users that are recently new will find the content new and relevant to them. In addition, there is benefits to bringing up a topic again as it will generate close to the same community interest as it did when it was initially posted. Our work could provide a way of detecting when might be the best possible time to repost something for it to be meaningful and relevant to the community.

An alternative application could be for buisness marketting. Knowing how a particular community will respond to your post can help buisness tailor their marketing strategies to a particular subreddit. There is also the possibility that future work on some of the proposed solutions/methods that we researched could help indicate things that a marketer could change in order to make their content generate more interest.

\section{Previous Work}
<NEED TO ADD TO THIS>

There are a number of websites dedicated to detecting ``reposts'' on Reddit, which are
exact duplicates of previously posted content.  One such website is KarmaDecay\footnote{\texttt{http://karmadecay.com/}}.
However, these websites only detect exact duplicates of hyperlinks and images, which make
up the majority of posts in many subreddits.  The textual content of posts in text-only
subreddits such as \texttt{/r/AskReddit} or \texttt{/r/ELI5} is not analyzed.
This gives these websites limited usefulness in text-only subreddits, where no images or external
links are generally allowed.

Furthermore, existing systems are designed mostly to help users avoid reposting exact duplicates of
existing material, and do not match based on a similarity measure.  This means that related or
very similar posts will not be matched, only exactly identical ones.  By using a text-matching system based
on similarity scores,
our project will allow text-only topics to be analyzed not to detect exact duplicates (as these
are rare in text posts due to the diversity inherently present in language), but to find similar
posts from the past based on matching keywords, topics, phrases, and other
content.  These similar posts can then be used to judge or estimate potential interest in the new
post and to predict useful information such as whether a post will be popular or successful or
which subreddits it may be most successful in by analyzing the interest in the previous posts. 
In addition, the uniqness of the post will also contribute to determing how popular a post might be.

Therre has been work done analyzing news content, not based on the article but
upon the responses of the general public \cite{liu:interest}. This paper describes
a way of what they describe as ``comment centric tagging'' as a means to identify
which articles were highly interesting and important. They use these comments within
a post to alter and adapt the interest weighting on an article. 

While this work is not the main focus of the paper, it does highlight how important
the comments of a post can be. The study they performed showed that the comments 
of a post greatly impacted the community interest within that specific topic. This will
help us direct or findings towards the comments found on Reddit, weighing them higher,
than the actual post itself.

Some work has also been done by researchers to determine characteristics of articles that
have been ranked by crowdsourced feedback from users \cite{askalidis:crowdsourced}.
The results of this study point out that while crowdsourced feedback mechanisms ``can be
relatively effective for \ldots promoting content of high quality, they do not perform well
for ordinal objectives such as finding the best articles.''  In other words, while interesting
topics and general themes are often highlighted as outstanding by user feedback, the best
specific articles are not always at the very top of the list.  Rankings which are based on
user feedback will likely tend to positively identify good
topics, but perhaps not the absolute best topics.  This means that we should not use a post's
score as an absolute ranking; that is, we should not conclude that one post is definitively better
than another simply because it has a higher score.  For our purposes, this means that
all posts with a reasonably high score could be determined to be interesting to the
community, but that comparing levels of interest (or ranking the level of interest)
between two interesting topics will likely be very difficult.

The article also has some
interesting discussion about crowdsourced scores over time, indicating that posts tend
to take a while to gain momentum, but then explode rapidly if they are determined to
be interesting.  This means that some posts which are not ``discovered'' by users (they
may be posted at the wrong time of day or when few users are online, so they are not
identified by interested users) may fail to gain traction and perform as they should.
This may mean that the same post, created at different times, may be extremely popular
one time and not at all another time.  In fact, the post may go unnoticed several times
and only be discovered once.  Our scoring algorithms will have to take this
into account, likely by giving the positive feedback of posts which are determined to
be interesting much more weight than the negative feedback of posts which are determined
to be uninteresting (because a post that scores highly is almost certainly
an indicator of community interest, while a post that scores poorly may have many
different reasons for receiving that score).

Work has been done in this area before by another Michigan student, targeting
Digg instead of Reddit. His
website\footnote{http://gigaom.com/2010/08/24/hey-digg-this-17-year-old-knows-what-you-are-thinking/}
analyzed all of the most recently-posted links on Digg, and used various
heuristics to predict which articles would make it to the front page. Various
differences between the algorithms used by the two sites make it impossible to
use the exact same approach here--the code relied on being able to see which
users voted on which articles, while on Reddit this information is made private
by default. In addition, Digg weighted the votes of some users above others
instead of treating everyone equally. The fundamental concept of extrapolating
early votes to predict eventual popularity will still be an important concept
for us to investigate.

In addition to using upvotes and downvotes to rank content, reddit keeps track
of the total score of all of a user's comments. Power users of reddit have a
strong incentive to identify posts that will eventually reach the front page,
since commenting early on these threads provides the maximum return of points
for the time spent. To this end, the subreddit \texttt{/r/risingthreads} tries
to identify these threads within minutes of when they are posted, but how it
does this is not known. The submissions to this subreddit are submitted by a
bot, that presumably flags posts with lots of early upvotes. In the sidebar of
the subreddit, \texttt{/r/askreddit} is specifically named as a subreddit for
which the bot's algorithm performs poorly. Our project, by analyzing the text
content of the post and its comments, may be able to improve upon this
performance. This subreddit also shows a way to get user feedback on our
project's effectiveness. If our project works well enough, we could set up our
own subreddit to share and test its results.

\section{Approach}
Our approach for this project involves three major areas: data collection, processing, and retrieval.

First, data must be collected from Reddit.  This will involve collecting and indexing a number of past posts
in text-only subreddits.  The posts and their comments will be analyzed, looking for common keywords,
themes, or topics which seem relevant to the post in order to match the post with other submissions
that have similar topics.

We focused on the ``score'' of a redit post (the number of upvotes minus the number of downvotes). When implementing our various different systems, we focused on the score as our measure of community interest. We thought of using the comments instead of the score, or inconjunction with the score, but we did not experiment and analyze that within our research and implementations.

There were three different implementations that we approached: cosine similarity between topics, naive bayseian on the words found within the posts, and a boolean classifier machine learning approach. We believed that each method had benefits and disadvantages and wanted to experiment with the various different methods to see which method, if any, peformed better than the baseline.

<cosine approach>

<keyword approach>

<machine learning approach>
Next, the popularity or interest of the post will then be measured through various statistics including
number of comments, number of upvotes/downvotes received (through Reddit's voting system), number
of distinct users who commented on the post, average length of the comments, average depth of
comment trees\footnote{On Reddit, users can comment on other comments, forming a tree-like structure of comments.},
number of upvotes/downvotes on comments and their distribution among the comments, time of day that the post was made,
and how long the post has been on Reddit.\footnote{Reddit's algorithm works in such a way that posts that have been on Reddit
a long period of time have a lower ``score''. We may have to inflate the score given to older posts as their ranking
has decayed over time.}

Finally, the keywords, and posts matching those keywords, will then be used to create a database which
can be used to judge popularity of a post based on which keywords appear in the post title or text
body.  This database can then be queried to determine expected popularity of a new post given its
text.  This database will also be designed in such a way that new posts on Reddit will be automatically
incoporated into the corpus.
The query system may also be able to suggest similar keywords (words which are frequently used with
the ones in the given text), topics, or information to include in the post which may increase
the post's popularity score (and thus hopefully lead to a more interesting post on Reddit).
\footnote{This is a possible extension of the project and will need to be explored further.}

\section{Evaluation}

<Nathan will fill this in>

\section{Results}

<keyword approach>

<cosine approach>

<machine learning approach>


% go to next column so it looks pretty and stuffs.
%\balancecolumns

\section{Conclusion}

<Nathan will fill this in>

%
% The following two commands are all you need in the
% initial runs of your .tex file to
% produce the bibliography for the citations in your paper.
\bibliographystyle{abbrv}
\bibliography{proposal}  % proposal.bib is the name of the Bibliography in this case
% You must have a proper ".bib" file
%  and remember to run:
% latex bibtex latex latex
% to resolve all references
%
% ACM needs 'a single self-contained file'!
%
%APPENDICES are optional
%\balancecolumns

%\subsection{References}
%Generated by bibtex from your ~.bib file.  Run latex,
%then bibtex, then latex twice (to resolve references)
%to create the ~.bbl file.  Insert that ~.bbl file into
%the .tex source file and comment out
%the command \texttt{{\char'134}thebibliography}.

% That's all folks!
\end{document}
