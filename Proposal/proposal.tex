\documentclass{acm_proc_article-sp}

\begin{document}

\title{Determining Community Interest in\\Text-Only Posts on Reddit}
\subtitle{Project Proposal}

%
% You need the command \numberofauthors to handle the 'placement
% and alignment' of the authors beneath the title.
%
% For aesthetic reasons, we recommend 'three authors at a time'
% i.e. three 'name/affiliation blocks' be placed beneath the title.
%

\numberofauthors{3}

\author{
% You can go ahead and credit any number of authors here,
% e.g. one 'row of three' or two rows (consisting of one row of three
% and a second row of one, two or three).
%
% The command \alignauthor (no curly braces needed) should
% precede each author name, affiliation/snail-mail address and
% e-mail address. Additionally, tag each line of
% affiliation/address with \affaddr, and tag the
% e-mail address with \email.
%
% 1st. author
\alignauthor
Charles Lewis\\
       \affaddr{University of Michigan}\\
       \affaddr{Ann Arbor, MI}\\
       \email{noodle@umich.edu}
% 2nd. author
\alignauthor
Nathan Price\\
       \affaddr{University of Michigan}\\
       \affaddr{Ann Arbor, MI}\\
       \email{nrprice@umich.edu}
% 3rd. author
\alignauthor
David Purser\\
       \affaddr{University of Michigan}\\
       \affaddr{Ann Arbor, MI}\\
       \email{dpurser@umich.edu}
}

\date{11 March 2014}

\maketitle
\begin{abstract}
This paper proposes a term project for the EECS 498 Information Retrieval
course at the University of Michigan which seeks to analyze content on 
Reddit\footnote{\texttt{http://www.reddit.com/}} to determine community
interest in certain topics, keywords, and questions.

The project involves collecting a set of data from a number of subreddits
(individual forums on Reddit) including textual content of each post,
number of votes that the post received, number of comments on the post, and
other information.  This information will be used to estimate the community's
interest in each post.  These estimates will be compiled and indexed by keyword,
allowing a user to query the system to determine the expected interest in a
new post and whether a new post is very similar to previous posts.
\end{abstract}

% A category with the (minimum) three required fields
\category{H.3.1}{Information Storage and Retrieval}{Content Analysis and Indexing}
\category{H.3.3}{Information Storage and Retrieval}{Information Search and Retrieval}
%A category including the fourth, optional field follows...
%\category{D.2.8}{Software Engineering}{Metrics}[complexity measures, performance measures]

%\terms{Algorithms, }

\keywords{Data analysis, Reddit, community interest} % NOT required for Proceedings

\section{Introduction}
Expanding on a slightly modified version of one of the suggested term projects,
this project aims to be able to accurately predict or measure community response
and interest in text-only posts (sometimes known as ``self-posts'') on the popular
internet forum Reddit.  These text-only posts are often found in dedicated ``subreddits''
such as \texttt{/r/AskReddit} or \texttt{/r/ELI5}.

The popularity of current and previous posts on Reddit can be estimated through the use
of comment counts, upvotes/downvotes\footnote{Users on Reddit can give either positive or
negative votes to posts and comments, known respectively as `upvotes' and `downvotes'.
Upvotes generally increase the visibility of a post or comment by causing it to be located
near the top of the page, whereas downvotes have the opposite effect.} given to the post by users, number of distinct
users who commented, average length of user responses, etc.

This popularity measure
can then be used to predict the expected popularity of a new question or topic by 
matching keywords or topics of the new post with those of past posts to determine
whether the new question is similar to any previously posted content, and, if so,
how popular the new post is expected to be.

\section{Previous Work}
There are a number of websites dedicated to detecting ``reposts'' on Reddit, which are
exact duplicates of previously posted content.  One such website is KarmaDecay\footnote{\texttt{http://karmadecay.com/}}.
However, these websites only detect exact duplicates of hyperlinks and images, which make
up the majority of posts in many subreddits.  The textual content of posts in text-only
subreddits such as \texttt{/r/AskReddit} or \texttt{/r/ELI5} is not analyzed.
This gives these websites limited usefulness in text-only subreddits, where no images or external
links are generally allowed.

Furthermore, existing systems are designed mostly to help users avoid reposting exact duplicates of
existing material, and do not match based on a similarity measure.  This means that related or
very similar posts will not be matched, only exactly identical ones.  By using a text-matching system based
on similarity scores,
our project will allow text-only topics to be analyzed not to detect exact duplicates (as these
are rare in text posts due to the diversity inherently present in language), but to find similar
posts from the past based on matching keywords, topics, phrases, and other
content.  These similar posts can then be used to judge or estimate potential interest in the new
post and to predict useful information such as whether a post will be popular or successful or
which subreddits it may be most successful in by analyzing the interest in the previous posts. 
In addition, the uniqness of the post will also contribute to determing how popular a post might be.

Therre has been work done analyzing news content, not based on the article but
upon the responses of the general public \cite{liu:interest}. This paper describes
a way of what they describe as ``comment centric tagging'' as a means to identify
which articles were highly interesting and important. They use these comments within
a post to alter and adapt the interest weighting on an article. 

While this work is not the main focus of the paper, it does highlight how important
the comments of a post can be. The study they performed showed that the comments 
of a post greatly impacted the community interest within that specific topic. This will
help us direct or findings towards the comments found on Reddit, weighing them higher,
than the actual post itself.

Some work has also been done by researchers to determine characteristics of articles that
have been ranked by crowdsourced feedback from users \cite{askalidis:crowdsourced}.
The results of this study point out that while crowdsourced feedback mechanisms ``can be
relatively effective for \ldots promoting content of high quality, they do not perform well
for ordinal objectives such as finding the best articles.''  In other words, while interesting
topics and general themes are often highlighted as outstanding by user feedback, the best
specific articles are not always at the very top of the list.  Rankings which are based on
user feedback will likely tend to positively identify good
topics, but perhaps not the absolute best topics.  This means that we should not use a post's
score as an absolute ranking; that is, we should not conclude that one post is definitively better
than another simply because it has a higher score.  For our purposes, this means that
all posts with a reasonably high score could be determined to be interesting to the
community, but that comparing levels of interest (or ranking the level of interest)
between two interesting topics will likely be very difficult.

The article also has some
interesting discussion about crowdsourced scores over time, indicating that posts tend
to take a while to gain momentum, but then explode rapidly if they are determined to
be interesting.  This means that some posts which are not ``discovered'' by users (they
may be posted at the wrong time of day or when few users are online, so they are not
identified by interested users) may fail to gain traction and perform as they should.
This may mean that the same post, created at different times, may be extremely popular
one time and not at all another time.  In fact, the post may go unnoticed several times
and only be discovered once.  Our scoring algorithms will have to take this
into account, likely by giving the positive feedback of posts which are determined to
be interesting much more weight than the negative feedback of posts which are determined
to be uninteresting (because a post that scores highly is almost certainly
an indicator of community interest, while a post that scores poorly may have many
different reasons for receiving that score).

Work has been done in this area before by another Michigan student, targeting
Digg instead of Reddit. His
website\footnote{http://gigaom.com/2010/08/24/hey-digg-this-17-year-old-knows-what-you-are-thinking/}
analyzed all of the most recently-posted links on Digg, and used various
heuristics to predict which articles would make it to the front page. Various
differences between the algorithms used by the two sites make it impossible to
use the exact same approach here--the code relied on being able to see which
users voted on which articles, while on Reddit this information is made private
by default. In addition, Digg weighted the votes of some users above others
instead of treating everyone equally. The fundamental concept of extrapolating
early votes to predict eventual popularity will still be an important concept
for us to investigate.

In addition to using upvotes and downvotes to rank content, reddit keeps track
of the total score of all of a user's comments. Power users of reddit have a
strong incentive to identify posts that will eventually reach the front page,
since commenting early on these threads provides the maximum return of points
for the time spent. To this end, the subreddit \texttt{/r/risingthreads} tries
to identify these threads within minutes of when they are posted, but how it
does this is not known. The submissions to this subreddit are submitted by a
bot, that presumably flags posts with lots of early upvotes. In the sidebar of
the subreddit, \texttt{/r/askreddit} is specifically named as a subreddit for
which the bot's algorithm performs poorly. Our project, by analyzing the text
content of the post and its comments, may be able to improve upon this
performance. This subreddit also shows a way to get user feedback on our
project's effectiveness. If our project works well enough, we could set up our
own subreddit to share and test its results.

\section{Approach}
Our approach for this project involves three major areas: data collection, processing, and retrieval.

First, data must be collected from Reddit.  This will involve collecting and indexing a number of past posts
in text-only subreddits.  The posts and their comments will be analyzed, looking for common keywords,
themes, or topics which seem relevant to the post in order to match the post with other submissions
that have similar topics.

Next, the popularity or interest of the post will then be measured through various statistics including
number of comments, number of upvotes/downvotes received (through Reddit's voting system), number
of distinct users who commented on the post, average length of the comments, average depth of
comment trees\footnote{On Reddit, users can comment on other comments, forming a tree-like structure of comments.},
number of upvotes/downvotes on comments and their distribution among the comments, time of day that the post was made,
and how long the post has been on Reddit.\footnote{Reddit's algorithm works in such a way that posts that have been on Reddit
a long period of time have a lower ``score''. We may have to inflate the score given to older posts as their ranking
has decayed over time.}

It is currently unclear whether these metrics will be adequate or accurate measures of community interest
in a post, but we assume that one or more of these measures in combination will be able to give insight
into the popularity of posts and comments based on certain topics or keywords.  For example, we assume that
posts with greater numbers of comments must have been more interesting or popular among users.  Likewise, we
assume that posts with large numbers of upvotes were likely interesting.  However, posts with many downvotes
present a less clear picture: the post may just be very controversial, but still interesting---or the post
may be badly written or uninteresting.  We will have to analyze the metrics to determine which of these
scenarios is more likely, given the other information we have collected about the post.

The various metrics may have different weighting as to analyzing the interest in the topic. As previously mentioned,
the numerical ``score'' seen on Reddit decays over time. Also, the upvote/downvote system can be abused by bots and
other humans attempting to abuse the system. Therefore, we believe that while the number of upvotes/downvotes will be 
useful, it will likely have a lower weight than the number of comments and the comparison to other, similar posts.



Finally, the keywords, and posts matching those keywords, will then be used to create a database which
can be used to judge popularity of a post based on which keywords appear in the post title or text
body.  This database can then be queried to determine expected popularity of a new post given its
text.  This database will also be designed in such a way that new posts on Reddit will be automatically
incoporated into the corpus.
The query system may also be able to suggest similar keywords (words which are frequently used with
the ones in the given text), topics, or information to include in the post which may increase
the post's popularity score (and thus hopefully lead to a more interesting post on Reddit).
\footnote{This is a possible extension of the project and will need to be explored further.}

\section{Testing}
We will have to analyze how our system performs in comparison to actual level of interest. To accomplish this, we will
provide the system with queries matching those of known range of documents. We will take a sample of topics found on Reddit
and have our system generate a interest score\footnote{We define our Interest Score as being an approximation of the number of
upvotes/downvotes, and comments that a query could expect to attain, or another similar measure. We may also include an ideal time of day to post, in addition
to a ranking on how it compares to all posts on a given subreddit (in the form of an Interest Percentage).} for them. 
We will then compare our generated interest score to the interest score of those in the sample.

We will make attempts to account for the time decay of upvotes/downvotes and the time of day the post was made. We will
also make sure to take a wide sample of posts, ranging from the most popular to least popular, to make sure that our system
is not biased to a specific subset of posts and manages to accurately predict an Interest Score which matches the
measured reality of those posts.


\section{Implementation}
We will implement the project in Python, using Python libraries and tools to download, analyze,
store, index, and query content from Reddit.

Reddit provides a web-based API\footnote{\texttt{http://www.reddit.com/dev/api}} for accessing its
content directly, eliminating the need for complicated scraping mechanisms and reducing server
load introduced by repetitive page generation during crawling.  This API appears to provide all
of the necessary functions that we would require for this project, such as the ability to
obtain posts, their upvote and downvote counts, comments, comment scores, information about
users, and many other useful pieces of information.

Reddit has some guidelines on their API usage\footnote{\texttt{http://github.com/reddit/reddit/wiki/API}},
and expects users to rate-limit their gathering of information to approximately one request every two seconds
(technically, thirty requests every minute).  Other restrictions on API usage, such as valid
and correct User-Agent strings which include version numbers, must be obeyed as well.
Care must be taken to properly use the API according to all of Reddit's rules.

There are pre-existing libraries for Python which can interface with the Reddit API, such as
the \textit{Python Reddit API Wrapper}\footnote{\textit{PRAW}, \texttt{https://praw.readthedocs.org/en/latest/}}.
The PRAW API is easy to use and is well-documented, and also follows all of Reddit's API rules
mentioned previously, making it easy to obtain data from Reddit without worrying about the raw
data processing.  This will give us more time to spend analyzing and retrieving information
from the databases we build.

Furthermore, Reddit's underlying server-side source code is also written
in Python and is available on GitHub\footnote{\texttt{https://github.com/reddit/reddit}}.  Analysis of this
code may be useful for determining the best uses of the statistics provided by Reddit.  For instance,
Reddit's own ranking algorithm which determines which posts should appear near the top of a given
subreddit uses the post's age and upvote/downvote counts in order to determine the post's position,
and analyzing this code may give us some useful insight into measures which we may want to use to
judge interest in posts.  It also may give us a baseline comparison point for recent posts.

Additionally, Reddit ``fuzzes'' the values it reports for upvotes/downvotes on currently popular posts, in
order to avoid vote manipulation by users.  Analyzing the source code for Reddit should allow us to
avoid retrieving incorrect data because of this issue.

Bots and users on Reddit commonly downvote old posts for several different reasons, which can skew
results.  It is unclear whether these downvotes create a significant impact on the total score of
posts in the long term, but care must be taken to watch for and account for these factors so that
our data is correct and accurate.

As ``community interest'' is a relatively vague and subjective term, it is difficult to define a strong
baseline comparison for measuring the accuracy of our algorithm.  However, we can use a number of techniques
to gauge relative accuracy.  For instance, being able to accurately predict, based on text alone, whether a new post will reach the front
page of a subreddit (top 25 of recent posts, according to Reddit's ranking algorithm) should be a good
start.  Similarly, the algorithm should assign high interest ratings to all posts which appear on
the front page, when supplied only with the text of the post, indicating agreement with the users of
Reddit and with Reddit's own algorithms.  Therefore, we could create snapshots
of the front page and of recently submitted posts at regular intervals, and use these to measure
the effectiveness of our algorithms by routinely calculating interest for all posts on the front page
and for all recently submitted posts.  We expect the posts on the front page to have a higher interest
rating than the random newly submitted posts.  Furthermore, by running different versions of the algorithm
on the same set of front page posts, we can update and tweak variables in the algorithm until it ranks these
posts more highly, indicating agreement with Reddit's own ranking algorithms.

Our project implementation will likely use a database storage system to store
gathered information permanently so that it can be re-analyzed, re-processed, and re-indexed as
our algorithms are developed and refined, without having to download all of the information again.
This will allow us to have a nearly continuously growing database from which to pull data from, so
our algorithms should get more and more accurate over time as they gain access to more and more
data.  However, we have not decided on a particular storage mechanism.

Using a complex database storage mechanism as opposed to simple data structures will allow us to more
easily do complex analysis of information.  For instance, summing the upvote/downvote totals for all
comments on a given post can be done in a single, simple SQL statement, whereas it would take several
loops to do this from Python.  On the other hand, a database storage engine adds complexity and can
be more difficult to manage.  An alternative would be to use Python's \texttt{pickle} library, which
stores native Python data structures (lists, tuples, and dictionaries) directly in byte-form into a file,
allowing them to be saved and loaded at any time.  However, depending on the amount of data collected,
it may be unreasonable to store all of the data in memory, or even in one file, and it may be necessary
to use additional file storage as well.

% go to next column so it looks pretty and stuffs.
%\balancecolumns

Some additional work will have to be done to create an interface for querying the database.  This
interface could be implemented in a terminal-based system for basic querying, or in a
browser-based setting (for example through Python's \texttt{SimpleHTTPServer} modules) to allow
for better data visualization of the results.  The relevant information to be displayed has not
yet been determined, so the implementation details of the user interface will be determined later,
but will likely include both the ability to estimate score of new posts and also to view scores
and other information about posts that have been indexed previously.


\section{Group Tasks and \\Responsibilities}

This project has many pieces that need to fit together properly to produce a working product.
As mentioned in prior sections, the main parts of the project are: gathering and storing data
from Reddit; developing algorithms for determining the interest in posts; matching new posts
with similar or related posts from the past; creating some sort of interface for
displaying the results to the user; and possible further extensions on this project.

These parts are all heavily dependent on one another, so it would be difficult to completely
split the work of the project among group members from the very beginning. Charlie and David
will be heavily focused in gathering, storing, and handling the database, querying, and API
usage. Charlie will also be focused on implementing our system on heroku and possibly
building a bot to actively gather and assess new data. Nathan will primarily focused 
on the ranking and anlyzing posts based on various factors, and will also help with the similar post
analysis. The similar post analysis will primarily driven by David. The display and output
of our system will be a collective work of all of us, and will the final remaining part of 
our project, besides the extention features.

While each of these tasks have been split up for the sake of listing what each project
member will be primarily focusing on, they are all heavily intergrated with one another
and will require each of us to have an in-depth understanding of how each part functions.
We will just specailize in one area more than others such that one person will have a greater
knowledge of there specific part. This will allow us to provide a better final presentation and
be able to contribute more to our collective team.


Once the main portion of the project is working, we can begin working on improving parts
separately as necessary to improve the algorithms, collection, and display of information
to the user.

All work besides the programming work will be a joint effort and worked on through various pair work resources
(Google Docs, git for this proposal management, etc.).


%
% The following two commands are all you need in the
% initial runs of your .tex file to
% produce the bibliography for the citations in your paper.
\bibliographystyle{abbrv}
\bibliography{proposal}  % proposal.bib is the name of the Bibliography in this case
% You must have a proper ".bib" file
%  and remember to run:
% latex bibtex latex latex
% to resolve all references
%
% ACM needs 'a single self-contained file'!
%
%APPENDICES are optional
%\balancecolumns

%\subsection{References}
%Generated by bibtex from your ~.bib file.  Run latex,
%then bibtex, then latex twice (to resolve references)
%to create the ~.bbl file.  Insert that ~.bbl file into
%the .tex source file and comment out
%the command \texttt{{\char'134}thebibliography}.

% That's all folks!
\end{document}
